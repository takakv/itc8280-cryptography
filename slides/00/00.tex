\title{ITC8280 Fundamentals of Cryptography}
\subtitle{Course organisation}
\date{\today}
\author{Taaniel Kraavi}
\institute%
{%
  \textit{School of Information Technologies}\\
  \textit{Tallinn University of Technology}
}

\begin{document}
\begin{frame}
  \titlepage
\end{frame}

\begin{frame}
  \frametitle{Course aim}

  \begin{columns}[T]
    \begin{column}{.5\textwidth}
      \textbf{Prerequisites:}
      \begin{itemize}
        \item Know basic programming
        \begin{itemize}
          \item We will use python3
        \end{itemize}
        \pause
        \item Know how to use git
        \pause
        \item Notions in:
        \begin{itemize}
          \item Discrete mathematics
          \item Binary data representation
          \item Group theory (bonus)
        \end{itemize}
      \end{itemize}
    \end{column}

    \pause
    \begin{column}{.4\textwidth}
      \textbf{Approach:}
      \begin{itemize}[<+->]
        \item Common terminology
        \item Cryptographic primitives as a black-box API
        \item Pitfalls
        \item DYOR: due diligence
      \end{itemize}
    \end{column}
  \end{columns}

  \vspace*{2em}

  \pause
  ITC8240 (master's course) for mathematical cryptography.\\
  \pause
  ITC8290 (master's course) for post-quantum cryptography.
\end{frame}

\begin{frame}
  \frametitle{Schedule}

  \textbf{Lectures}

  Tuesdays: 15:15--16:45 room ICT-315 (recorded)

  \pause
  \textbf{Practices}

  Wednesdays: 18:00--18:45 room ICT-315 (bring your laptops, not recorded)

  \vspace*{1em}

  \pause
  Important dates
  \begin{itemize}[<+(1)->]
    \item April 12 at 23:59 local time: midterm deadline
    \item May 10 at 23:59 local time: deadline for all homework
    \item May 20: first written exam attempt (provisional)
  \end{itemize}
\end{frame}

\begin{frame}{Grading}
  \textbf{Grade distribution}
  \begin{itemize}
    \item 40\% --- homework
    \item 10\% --- Cryptohack
    \item 50\% --- final exam
  \end{itemize}

  \vspace*{1em}

  \pause
  \begin{tabular}{cccccc}
    $\le 51$p or missed requirements & $[51, 60]$ & $[61, 70]$ & $[71, 80]$ & $[81, 90]$ & $[91, 100]$\\
    \midrule
    0 & 1 & 2 & 3 & 4 & 5
  \end{tabular}
\end{frame}

\begin{frame}
  \frametitle{Grading}

  \textbf{Homework}
  \begin{itemize}[<+(1)->]
    \item Up to 5 assignments
    \item Theoretical part (research) \& practical part (programming)
    \item Soft deadline: 2 weeks to complete
    \begin{itemize}
      \item You can resubmit an improved version
      \item Assignment grade: original grade + 50\% of fixes grade
    \end{itemize}
    \item Hard deadline: end of week 14 (May 10)
    \begin{itemize}
      \item Grade is final
    \end{itemize}
    \item Homework is easier if you follow practice sessions!
  \end{itemize}

  \pause
  Course pass requirement: $\ge 20$ points from homework.
\end{frame}

\begin{frame}
  \frametitle{Grading}

  \textbf{Midterm}
  \begin{itemize}
    \item Mandatory to attempt (need to score at least one point)
    \item Remote, on Moodle
    \item Reality check for your knowledge
    \item No effect on final grade
  \end{itemize}

  \vspace*{1em}

  \pause
  \textbf{Cryptohack}
  \begin{itemize}
    \item Online CTF-style platform to learn crypto basics
    \item Worth 10 points
    \item Grade $=$ completion percentage
  \end{itemize}
\end{frame}

\begin{frame}
  \frametitle{Grading}

  \textbf{Exam}
  \begin{itemize}[<+(1)->]
    \item 2h
    \item Closed book \& written
    \begin{itemize}
      \item You can bring a one-sided A4 of \emph{handwritten} notes
      \item Alternatively: 30min oral exam (max 10 slots), no cheat-sheet
    \end{itemize}
    \item Must have passed the midterm
    \item Must have at least 20 points from homework tasks
    \item Must pass ($\ge$ 3p) each exam section to pass the exam (T/F + 6 sections)
    \item Must score at least 1/3 of total exam points
  \end{itemize}
\end{frame}

\begin{frame}{Grading}
  \textbf{Extra credit}
  \pause
  \begin{itemize}
    \item There may be bonus assignments
    \begin{itemize}
      \item These count towards the homework grade
      \item Saturation is therefore possible: you cannot get over 40p for homework
    \end{itemize}
    
    \pause
    \item If missing $\le$ 3 points (out of 100) for a higher final grade:
    \begin{itemize}
      \item Personal extra credit might be possible, but guaranteed
      \item The possibility and tasks will depend on your overall diligence
      \item E.g. timely submissions, homework quality, discussions
    \end{itemize}
  \end{itemize}

  \vspace{1em}

  \pause
  If you do not put effort into the course, do not expect handouts.
  \begin{itemize}
    \item E.g. no post-exam extra credit if you haven't submitted some homework
  \end{itemize}
\end{frame}

\begin{frame}
  \frametitle{Grading}

  \textbf{Summary}
  \begin{itemize}[<+(1)->]
    \item Homework (40p, of which 20p are necessary)
    \item Midterm (ungraded, but necessary)
    \item Cryptohack (10p)
    \item 120min written exam (50p) with minimal pass thresholds
  \end{itemize}

  \vspace*{1em}

  \pause
  Not hard to pass, but requires consistent focus and effort.
  \begin{itemize}[<+(1)->]
    \item Do not let material pile up!
    It gets overwhelming quickly
    \item Questions --- they are essential to learning, so ask many!
    \item Make good use of the Discord server: ask questions and discuss
  \end{itemize}
\end{frame}

\begin{frame}{AI use}
  LLMs can make learning easier, but also much harder
  \begin{itemize}
    \pause
    \item LLMs feed into overconfidence. Beware!
    \begin{itemize}
      \item Cognitive gap between what you feel you understand, and what you can explain
    \end{itemize}
    \pause
    \item If you use LLMs in any assignment, you must always indicate its use
    \begin{itemize}
      \item This applies to rewording also
      \item LLMs can inject and remove meaning that you did not know/intend
    \end{itemize}
    \pause
    \item Forbidden uses in this course:
    \begin{itemize}
      \item Do not generate task-related code (or text, formulas, \dots)
      \item E.g. fine to ask it how to loop in Python, but not to write a loop for CBC
      \item Do not use at AI all if the task explicitly forbids its use
      \item Unreferenced AI use: if I catch you, it will be \enquote{ai ai}
    \end{itemize}
  \end{itemize}
\end{frame}

\begin{frame}
  \frametitle{Syllabus}

  \begin{columns}[t]
    \begin{column}{.5\textwidth}
      \begin{enumerate}
        \item Introduction
        \item Symmetric cryptography
        \item Symmetric cryptography (cont.)
        \item Hash functions \& MACs
        \item Public-key cryptography
        \item Public-key cryptography (cont.)
        \item Digital signatures \& certificates
        \item Public-key infrastructure
      \end{enumerate}
    \end{column}

    \begin{column}{.5\textwidth}
      \begin{enumerate}\setcounter{enumi}{8}
        \item eIDAS, ASiC-E, CDOC
        \item TLS, HTTPS, SSH
        \item Secure programming \& hardware
        \item Commitments \& ZKPs
        \item Secret sharing \& MPC
        \item Internet voting
        \item Post-quantum cryptography
        \item Review
      \end{enumerate}
    \end{column}
  \end{columns}

  \vspace*{1em}

  \pause
  Topics might not be cleanly split by week \& the order/grouping might change.
\end{frame}

\begin{frame}
  \frametitle{Reading materials}

  The book \enquote{The Joy of Cryptography} by Mike Rosulek is freely available at \url{https://joyofcryptography.com} and covers many of the topics in this course.

  \vspace*{1em}

  The book is more mathematical than this course, but can hence provide you with valuable insight and a stronger understanding of the same concepts. 
\end{frame}

\end{document}
