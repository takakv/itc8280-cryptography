\documentclass{homework}

\usepackage{enumitem}
\usepackage{tcolorbox}

\usepackage{tikz}
\usetikzlibrary{positioning,calc}

\newcommand*{\ENC}{\mathsf{Enc}}

\usepackage{listings}

\lstset{
basicstyle=\small\ttfamily,
columns=flexible,
breaklines=true
}

\usepackage{fancyvrb}
\fvset{listparameters=\setlength{\topsep}{0pt}\setlength{\partopsep}{0pt}}

\title{01}

\begin{document}
\maketitle

\textbf{Organisational details}

\begin{itemize}
  \item Include in your report approximately how long the homework took you.
  \item You must follow the submission instructions outlined on GitLab.
  \item If the GitLab pipeline does not succeed, I will not grade your assignment (you will receive 0).
  The same applies if the Markdown report is illegibly formatted or if you miss the final submission deadline.
  \item I will provide feedback on Discord and/or GitLab.
  Check both after receiving your grade.
  \begin{itemize}
    \item My initial feedback may be quite minimal as there are so many of you.
    However, you can request additional feedback.
    Specific questions are also welcome.
    \item Do not share the feedback you receive, as this may give others an unfair advantage.
  \end{itemize}
  \item Grace clause submission deadline: 07.03.2026 23:59 EET.
  \begin{itemize}
    \item The grace clause applies only if you submit the full work (all tasks) by the soft deadline and the GitLab pipeline succeeds.
    \item You have one week from receiving the feedback to fix and resubmit your improved work.
    You may regain up to (but not necessarily) half of the lost points.
  \end{itemize}
  \item You may share hints with each other, but not answers or code.
  Ask questions in the server if they could benefit everyone.
  \item You may -- and I encourage you to -- ask me questions.
  \item You may use AI as assistance, but not to solve the tasks themselves.
  \begin{itemize}
    \item Acceptable: e.g. checking your understanding, asking how to loop in Python.
    \item Not acceptable: e.g. generating code logic, library use, or crypto operations.
    \item If you use AI in any form, including for rewording, state in your report where and what you used it for.
    \item I recommend asking in the server or contacting me instead of using AI.
  \end{itemize}
  \item If I catch you plagiarising, copying, or using AI for code logic, I will report you to the faculty and fail you for the course.
  \begin{itemize}
    \item If during the exam you are unable to explain material for which you received homework credit, I may ask you to explain your work.
    \item If you cannot convince me that the homework is your own work, I will report you to the faculty and fail you.
  \end{itemize}
\end{itemize}

\newpage

\begin{center}
  Theory tasks
\end{center}

\textbf{Additional instructions}

\begin{itemize}
  \item \textbf{Please do not use AI for the theoretical part.}
  Challenge your understanding and look for additional written sources if the lecture and provided materials are not enough.
  Critical thinking and clear communication are skills -- use them or lose them.
  \item You must reference all external materials you rely on, including AI tools, if you choose to use them anyway.
\end{itemize}

\begin{task}{A little about you}
  What role do you currently work in, or hope to work in?
  Is cryptography relevant to your work, or do you expect it to become relevant?
  What do you hope to gain from this course?

  \textit{Just a few sentences (3--4) is fine.}
\end{task}

\begin{task}{A choice}
  Select one of the two tasks below.

  \textit{Task A.}
  What programming language do you use most / are you most comfortable with?
  For that language, list the following:
  \begin{enumerate}
    \item The standard library / namespace / module (if any) for cryptographic randomness.
    \item The standard library / namespace / module (if any) for encryption.
    \item Three well-regarded third-party libraries for cryptographic functionality (any kind).
    \item One cool cryptographic library, tool, or project not already listed.
  \end{enumerate}

  \textit{Task B.}
  Benchmark how long it takes for your computer to iterate over the 32-bit space.
  Based on that, estimate how long it would take to cover the 128-bit space.

  Show your calculations and give the final result in seconds and years.
\end{task}

\begin{task}{A pledge}
  \enquote{%
    I understand that I should not implement my own cryptography for any production system, i.e. for any purpose other than personal learning.
    I also understand that any implementations done during this course are for learning purposes only and are not for real-world use.
    I understand that due diligence is crucial.%
  }

  \textit{Answer with yes/no.}
\end{task}

\newpage

\begin{task}{RNGarök}
  Search for one example of a real-world RNG failure and its consequences.
  In a few short sentences, state:
  \begin{itemize}
    \item the system or product,
    \item what went wrong,
    \item what the impact was, and
    \item what should have been done instead.
  \end{itemize}

  Include your source(s).
  It does not have to be a research article; a popular science blog or news post is fine.
\end{task}

\begin{task}{Another Encryption Standard}
  Skim the public comments submitted in response to the call for comments on \href{https://csrc.nist.gov/pubs/sp/800/197/iprd}{NIST SP 800-197}.
  You do not need to read all comments in full; it is fine to focus on only a few.

  Identify and briefly explain, in your own words:
  \begin{itemize}
    \item two potential advantages of standardising and adopting Rijndael-256, and
    \item two potential disadvantages.
  \end{itemize}

  The goal of the task is to familiarise you with the \enquote{Call for Comments} process.
  The backgrounds and expertise of authors vary, which is why the process is important.
  The same is true for readers, and comments can be useful even if you do not understand some (or even most) parts.
\end{task}

\newpage
\setcounter{task}{0}

\begin{center}
  Practical tasks
\end{center}

\textbf{Additional instructions}

\begin{itemize}
  \item I must be able to run your program with Python 3.14 and OpenSSL v3.6.1\footnotemark{}.
  \footnotetext{You may use older versions of Python and OpenSSL as long as they are interoperable with mine.}
  \item The GitLab pipeline must succeed before you submit your assignment.
  \item You may not import any third-party module other than \texttt{cryptography}.
  You are free to import any built-in module.
  If you feel the need to import another third-party module, please check with me beforehand!
  \item All errors must be handled: print a descriptive message starting with \enquote{Handled:} and exit with error code \texttt{1}.
  Filesystem errors (e.g. file read/write errors) need not be handled.
  \item You must identify potential edge cases yourself.
  You may not ask an AI to implement them or identify them for you.
  \item You may change anything in the template files, unless specified otherwise, as long as the GitLab pipeline succeeds.
\end{itemize}

\begin{task}{ECB CPA}
  Task template file: \texttt{ecb\_cpa\_template.py}

  Implement the function
  \begin{Verbatim}
challenger(m0: bytes, m1: bytes) -> bytes | None
  \end{Verbatim}
  This function must implement the IND-CPA challenger for AES-ECB with 256-bit keys.
  For each call, the challenger uniformly selects either \texttt{m0} or \texttt{m1}, securely generates a fresh key, encrypts the selected message, and returns the ciphertext.

  If the messages do not satisfy the requirements for the (symmetric) IND-CPA game, or any other functional requirements, the challenger aborts the game by returning \texttt{None}.

  Implement the function
  \begin{Verbatim}
distinguisher() -> bool
  \end{Verbatim}

  The function must implement the IND-CPA adversary.
  The adversary does not have access to the challenger's encryption key or its choice, and must be able to win the game with a single challenger call.
  
  The adversary returns \texttt{True} if it considers \texttt{m1} to be encrypted, \texttt{False} otherwise.
\end{task}

\newpage

\begin{task}{CBC CCA}
  Task template file: \texttt{cbc\_cca\_template.py}

  Implement the function
  \begin{Verbatim}
distinguisher() -> bool
  \end{Verbatim}

  The function must implement the IND-CCA2 adversary for AES-CBC with 128-bit keys.
  The adversary does not have access to the challenger's encryption key or its choice.
  
  The adversary returns \texttt{True} if it considers \texttt{m1} to be encrypted, \texttt{False} otherwise.

  Write in the report how you came up with your attack.
  Include also brief descriptions of ideas that you considered or tried, but which did not work.
  E.g. you can make a chronological list of ideas.
  I recommend that you write them down as you go, rather than try to recall them in the end.
\end{task}

\begin{task}{Twice the fun}
  Task template file: \texttt{chachatwo\_template.py}\\
  Supporting file: \texttt{ccpoly\_encrypt.py}

  \texttt{chachatwo.py} is a program that takes as input two hex-encoded ChaCha20-Poly1305 ciphertexts.
  Assuming that they were encrypted with the same key and tag, it recovers the original plaintexts with the help of a wordlist.
  Implement it based on the template.

  The wordlist is assumed to be a plaintext file where each row contains one word.

  Use the program to recover the two words from the authenticated ciphertexts:
  \begin{itemize}
  \item \texttt{e1d193c3e71e5bdee35b0fb9a129ca76de64a57f8d}
  \item \texttt{e0df8bcff162933d5edbbb6c5204763ddb0b6805e1}
  \end{itemize}
  The original words are common English words.
  Include them both in your report.

  Can you tell which word corresponds to which ciphertext?
  Why yes/why not?
\end{task}

\end{document}
