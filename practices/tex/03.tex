\documentclass{practice}
\usepackage{verbatim}

\title{3}
\date{\today}

\begin{document}
\maketitle

\begin{task}{AES test vectors}
  During the lectures, we learnt that AES is standardised by NIST.

  NIST has a program called the \href{https://csrc.nist.gov/projects/cryptographic-algorithm-validation-program/block-ciphers}{\enquote{Cryptographic Algorithm Validation Program}} (CAVP) and part of it are testing specifications and \emph{test vectors} for block ciphers.

  Fetch some test vectors from their website, and use them to test OpenSSL's AES implementation.
  Recall that in practice, you should never provide the encryption key/password directly on the CLI in a manner which saves it into the CLI history.
\end{task}

\begin{task}{Your personal Tux}
  For this task, you will need \href{https://imagemagick.org/}{\texttt{imagemagick}}.

  First, fetch \href{https://taltech.ee/brand}{\textit{TalTech's logo}}, or some other image file of your choice.
  The image file should be of a fairly high resolution, preferably of at least 1000px in in both dimensions, and contain large sections of the same colour.

  For practicality, convert the logo to BMP, a lossless data format, with:
  \begin{Verbatim}
magick logo.jpg bmp3:logo.bmp
  \end{Verbatim}

  Encrypt the image with AES in ECB mode with OpenSSL.

  Before you can visualise the image, you must restore the now encrypted 54-byte BMP header, otherwise the OS / file viewer does not know how to open/display the image.
  You can do this using the \texttt{dd} tool to overwrite the starting bytes of the encrypted image.
  \begin{Verbatim}
dd if=logo.bmp of=logo.ecb.bmp bs=54 count=1 conv=notrunc
  \end{Verbatim}

  Try the same thing with CBC and CTR mode.
  Visualise the images.

  Finally, generate an image of identical dimensions from crypto-random data.
  For this, first get the original image's filesize:
  \begin{Verbatim}
FILESIZE=$(wc -c < logo.bmp | tr -d '[:space:]')
  \end{Verbatim}

  Generate that many random bytes with OpenSSL:
  \begin{Verbatim}
openssl rand -out random.bmp "$FILESIZE"
  \end{Verbatim}

  Finally, replace the header as previously:
  \begin{Verbatim}
dd if=logo.bmp of=random.bmp bs=54 count=1 conv=notrunc
  \end{Verbatim}
\end{task}

\begin{task}{A doc a day keeps mistakes away}
  Visit \href{https://cryptography.io/en/latest/}{\textit{pyca/cryptography}}'s documentation page.
  Identify the modules which allow you to use symmetric encryption, authenticated symmetric encryption, block cipher modes of operation, and symmetric padding.

  Pay close attention to the various warnings and notices.
  What do you think, should the \enquote{Hazardous Materials} module warning apply to you?
\end{task}

\end{document}
